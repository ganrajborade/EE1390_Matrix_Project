\documentclass{beamer}
\usepackage{graphicx}
\usepackage{nicefrac}
\usetheme{Warsaw}
\graphicspath{ {Home/Desktop/} }
\usepackage{tikz}
\usepackage{tkz-euclide} % loads  TikZ and tkz-base
\usetkzobj{all}
\title[]{EE1390}
\subtitle{Matrix Project}

\institute{IIT HYDERABAD}
\author{Ganraj(EE18BTECH11016) and Kartikeya(EE18BTECH11025)}

\date{}

\begin{document}


\begin{frame}
\titlepage
\end{frame}

\begin{frame}{Question}

The equation of perpendicular bisectors of sides AB and AC of a triangle ABC  are x-y+5=0 and x+2y=0 respectively. If a point A is (1,-2) then the equation of line BC is 





\setlength{\parindent}{8cm}
[IIT 1986]
\end{frame}

\begin{tikzpicture}
[
 scale=0.9,
  >=stealth,
  point/.style = {draw, circle, fill = black, inner sep = 1pt},
]

\node (A) at (1,-2) [point,label = below right:$A{(1,-2)}$] {};
\node (B) at (-7,6)[point,label = above left:$B{}$] {};
\node (C) at (2.2,0.4)[point,label = above right:$C{}$] {};
\node (Y) at (1.6,-0.8)[point,label = above right:$Y{}$] {};
\node (N) at (-6,3)[point,label = above left:$N{}$] {};
\node (X) at (-3,2)[point,label = above right:$X{}$] {};
\node (M) at (-5,0)[point,label = below right:$M{}$] {};
\node (Z) at (-10/3,5/3)[point,label = below:$Z{}$] {};






\draw (A) -- (B) -- (C) -- (A) -- (Y) -- (N);
\tkzMarkRightAngle[fill = blue!20,size=.2](A,Y,Z)
\tkzMarkRightAngle[fill = blue!20,size=.2](B,X,Z)
\draw (A) -- (B) -- (C) -- (A) -- (X) -- (M);


\end{tikzpicture}

 


\begin{frame}{Solution(Approach 1)}
Converting the problem in matrix form:




\vspace{2 mm}
Let the perpendicular bisector of AB and AC are XZ and ZY respectively.



Equation of XZ and ZY in matrix form are respectively:
\setlength{\parindent}{4cm}  

     
 (1 \ -1)$\boldsymbol{x}+5=0$

		    (1 \  2)$\boldsymbol{x}=0$
\setlength{\parindent}{0cm}


Since the equation in matrix form of line is
\begin{align*}
n^Tx = p; 
\end{align*}
\vspace{2 mm}
where n is normal vector to the line.


\vspace{3 mm}
Hence the normal vector of XZ = $\binom{1}{-1}$ and the normal vector of ZY = $\binom{1}{2}$

\end{frame}

\begin{frame}
\vspace{2 mm}
So The direction vector of $m_1$ of XZ is $\binom{1}{1}$
\vspace{2 mm}
and the direction vector of $m_2$ of XZ is $\binom{-2}{1}$.
\vspace{2 mm}
Hence, the equation of AB matrix form is : 
\vspace{2 mm}

\setlength{\parindent}{3.6cm}
(1 \ 1)$\boldsymbol{x}$=$p_1$
\vspace{2 mm}

\setlength{\parindent}{0cm}
Now since AB passes through A=$\binom{1}{-2}$, so
\vspace{2 mm}

\setlength{\parindent}{3.6cm}
(1 \ 1)$\binom{1}{-2}$= $p_1$
\vspace{2 mm}


$p_1$ = -1
\setlength{\parindent}{0cm}
\vspace{2 mm}


Therefore equation of AB is (1 \ 1)$\boldsymbol{x}=-1$


\vspace{2 mm}
Similarly, the equation of AC matrix form is : 
\vspace{2 mm}

\setlength{\parindent}{3.6cm}
(-2 \ 1)$\boldsymbol{x}$=$p_2$
\vspace{2 mm}

\setlength{\parindent}{0cm}
Now since AC passes through $\binom{1}{-2}$, so
\vspace{2 mm}

\setlength{\parindent}{3.6cm}
(-2 \ 1)$\binom{1}{-2}$= $p_2$
\vspace{2 mm}


$p_2$ = -4
\setlength{\parindent}{0cm}
\vspace{2 mm}


\end{frame}

\begin{frame}
\vspace{2 mm}
\setlength{\parindent}{2cm}
Therefore equation of AC is (-2 \ 1)$\boldsymbol{x}=-4$
\vspace{2 mm}




\setlength{\parindent}{0cm}

Now Let's find out the point of intersection of AB and XZ

\setlength{\parindent}{3.6cm}
\vspace{2 mm}
(1 \ -1)$\boldsymbol{x} = -5$    \ \ (XZ)

\vspace{2 mm}
(1 \ 1)$\boldsymbol{x}=-1$           \ \ (AB)

\vspace{2 mm}
 \setlength{\parindent}{0cm}
  Stacking both equations:
  \setlength{\parindent}{3.6cm}
  \vspace{2 mm}
   $\binom{1 \ -1}{1 \ 1}$ $\boldsymbol{x}$ = $\binom{5}{-1}$

   \vspace{2 mm}
$\boldsymbol{x}$ = $\binom{-3}{2}$

\setlength{\parindent}{0cm}
Similarly,the point of intersection of ZY and AC:

\setlength{\parindent}{3.6cm}
(-2 \ 1)$\boldsymbol{x}$ = -4          \ \ (AC)

\vspace{2 mm}
  (1 \  2)$\boldsymbol{x}=0$                 \ \ (ZY)
  
\vspace{2 mm}
  \setlength{\parindent}{0cm}
  Stacking both equations:
  \setlength{\parindent}{3.6cm}
  \vspace{2 mm}
  $\binom{1 \ 2}{-2 \ 1}$ $\boldsymbol{x}$ = $\binom{0}{-4}$
  \vspace{2 mm}
  
  
$\boldsymbol{x}$ = $\binom{1.6}{-0.8}$
\end{frame}

\begin{frame}

\vspace{2 mm}
By mid point formula  X = (A+B)/2 and Y = (A+C)/2


\vspace{2 mm}
so, B = 2X - A and C = 2Y - A


\vspace{2 mm}
B = 2$\binom{-3}{2}$ - $\binom{1}{-2}$ and C = 2$\binom{1.6}{-0.8}$ - $\binom{1}{-2}$
  \setlength{\parindent}{1.2cm}
  
  
\vspace{2 mm}
B = $\binom{-7}{6}$ \ and \ C = $\binom{2.2}{0.4}$
  \setlength{\parindent}{0cm}
\vspace{2 mm}


Hence the direction vector of BC is $\binom{9.2}{-5.6}$\


\vspace{2 mm}
so normal vector is $\binom{-5.6}{-9.2}$\


\vspace{2 mm}
Equation of BC is $n^T$ $\boldsymbol{x}$ = $n^T$B 



\vspace{2 mm}
or it can be also written as  $n^T$ $\boldsymbol{x}$ = $n^T$C.
For simplicity in calculation,we are considering B.


\vspace{2 mm}
  \setlength{\parindent}{3.6cm}
$n^T$ = (-5.6  \ -9.2)  .
Because  n = $\binom{-5.6}{-9.2}$\

  \setlength{\parindent}{0cm}
\vspace{2 mm}
So equation of BC is (-5.6  \ -9.2)$\boldsymbol{x}$ = -16
  \setlength{\parindent}{3.6cm}
\vspace{2 mm}


(14  \ 23)$\boldsymbol{x}$ = 40
\end{frame}

\begin{frame}{Approach 2(Generalized way)}
\vspace{2 mm}
If we are given a line (m \ -1)$\boldsymbol{x}$ = -c, then the mirror image of a point of a point P=$\boldsymbol{x}$ is given by

\vspace{2 mm}

$\boldsymbol{x'}$  = \nicefrac{1}{(1+ $m^2$)} $\binom{1 - m^2 \ \ 2m}{\ 2m \ \ m^2 - 1}$ $\boldsymbol{x}$ -  \nicefrac{1}{(1+ $m^2$)} $\binom{2mc}{-2c}$
\vspace{2 mm}

In our case, image of A about  (1 \ -1)$\boldsymbol{x}$ = -5 is B.
\vspace{2 mm}
(Because we are taking the image across the perpendicular bisector)




\vspace{2 mm}
And image of A about  (1 \ 2)$\boldsymbol{x}$ = 0 is C.



\vspace{2 mm}
So for (1 \ -1)$\boldsymbol{x}$ = -5, m=1 and c=5,Hence



\vspace{2 mm}
$\boldsymbol{x'}$  =  $\binom{0 \ \ 1}{\ 1 \ \ 0}$ $\binom{1}{-2}$ -   $\binom{5}{-5}$


\vspace{2 mm}
$\boldsymbol{x'}$ = $\binom{-7}{6}$


\vspace{2 mm}
Therefore,B = $\binom{-7}{6}$
\end{frame}
\begin{frame}
\vspace{2 mm}
Similarly for  (1 \ 2)$\boldsymbol{x}$ = 0, m=-1/2 and c = 0,Hence


\vspace{2 mm}
$\boldsymbol{x'}$  =  $\binom{3/5 \ \ -4/5}{\ -4/5 \ \ 3/5}$ $\binom{1}{-2}$ -   $\binom{0}{0}$


\vspace{2 mm}
$\boldsymbol{x'}$ = $\binom{11/5}{2/5}$

\vspace{2 mm}
Therefore,C = $\binom{11/5}{2/5}$


\vspace{2 mm}
Since we have the two coordinates B and C, 
the equation of BC is 

\setlength{\parindent}{3.6cm}
\vspace{2 mm}
(14  \ 23)$\boldsymbol{x}$ = 40
\end{frame}
\includegraphics[scale=0.7]{Figure.png}

\end{document}
